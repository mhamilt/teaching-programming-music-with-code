\section{Introduction}
% What is this chapter (very brief)?

This chapter is designed to serve as a brief tour of connections between music and programming. In particular, it is aimed at educators looking to explore music in their pedagogy.

The following sections explore motivations for connecting music with programming (Section \ref{sec:why}), the rich existing work in this domain (Section \ref{sec:literature}), and present a set of accompanying example resources that can serve as starting points into different kinds of music/coding projects (Section \ref{sec:resources}).



% Music and computers have a long history together. Musicians have been making sounds by programming computers since the early 1950s (CITATIONS), both to synthesise digital sounds and to explore new approaches to composition. 

There are many ways in which sound and music may connect with programming. As an indication of the possible variety, here is a non-exhaustive list:

\begin{itemize}
    \item composing with code: using code to write pieces of music
    \item performing with code: writing code in a performance to make music
    \item creating instruments: building software instruments to subsequently perform with
    \item understanding sound: coding to analyse and explore representations of sound and/or music
    \item data sonification: representing existing data as sound or music, either in real-time or with  historical data
    \item ...?
\end{itemize}



\section{WHY: why is music of any value in a programming and education context} \label{sec:why}

% (from the pitch)
Music and programming have a rich intertwined history. Simple-but-effective connections can be found between programming concepts (iteration, abstraction, conditionals, loops) and musical concepts (timbre, rhythm, timing, harmony, pattern). Writing a program that generates sound provides an immediate, engaging way to manifest abstract processes; in a pedagogical sense, the sound output is a form of feedback. Programs can often be short and simple but nevertheless very musically satisfying. 



It's not abstract! You get something at the end of it that you can show people.

(some useful lines in the pitch on this)

interesting parallels with music and code anyway: scores are instructions too.

Rhythm and pitch - tailor made for iteration, conditionals, etc.

Music to teach/engage with simulation (Mike, Charlotte). Fourier transform as a nice transferable example...

Examples:
\begin{enumerate}
    \item Fourier transform
    \item Wave interference
    \item Resonance cavities $\leftrightarrow{}$ Guitars
    \item Oscillators, coupled oscillators (Idea of pendulums that make sound. Need to increase corresponding frequency to make it audible.)
\end{enumerate}


\section{Overview of what exists already} \label{sec:literature}
%Also quite brief, but pointing towards lots of existing work around
 %- introducing people to programming through music
% - introducing people to music through programming

Some relevant references:
\begin{enumerate}
\item Creativity in CS1: A Literature Review \cite{Sharmin2021}
\item Sonic Pi – performance in education, technology and art \cite{Aaron2016}
\item \href{https://www.raspberrypi.org/app/uploads/2021/11/Teaching-programming-in-schools-pedagogy-review-Raspberry-Pi-Foundation.pdf}{Waite, Jane, and Sue Sentance. "Teaching programming in schools: A review of approaches and strategies." Raspberry Pi Foundation (2021).}
\item Live Coding book - Blackwell et al 2022
\item Earsketch, e.g. Freeman et al 2019
\item Music in R: \href{https://flujoo.github.io/gm/articles/gm.html}{https://flujoo.github.io/gm/articles/gm.html}
\item Benefits of integrating arts into STEM (STEAM): \href{https://doi.org/10.1016/j.procs.2013.09.317}{https://doi.org/10.1016/j.procs.2013.09.317}
\end{enumerate}

 This section is primarily about where education, music and programming come together.


 There is a substantial musical live coding community that serves as an access point to programming for many, as well as an access point for music for some programmers.

A wide variety of bespoke languages and tools have sprung up in relation to this, often building on top of the SuperCollider environment (CITE). Sonic Pi is particularly notable for being explicitly designed to support the teaching of programming \textbackslash{}cite\{Aaron16\}, particularly in schools for younger students. Live coding for music is often taught in a short workshop context \textbackslash{}cite\{Blackwell2022\}).

The EarSketch project \textbackslash{}cite\{Freeman19\} is similar in it’s pedagogical aims, in that it seeks to broaden participation in computing by demystifying coding and relating it to a domain of interest to students. The project does this by not only including a simplified language for coding, but by integrating the coding into a digital audio workstation framework that runs in the browser: a familiar workspace for those with musical backgrounds. 

A key concern across all the above projects is that there should be room for musical variety and expression; although they may present simple entry points to engaging with programming, it is possible to create rich and interesting creative work that students will be proud of.


 

\section{Overview of our online resources} \label{sec:resources}
%What are they, who are they for, what kinds of programming concepts do they relate to?

% Our working list of projects is here:
% https://docs.google.com/document/d/1FsmVMrJlf2lRL4Jb25oat9y6M3Uvrkz7JEuytC6oyb8/edit?usp=sharing

\section{Mini summary}
Close it with a quick summary

