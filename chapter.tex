\section{Introduction}
What is this chapter (very brief)?

\section{Overview of what exists already}
Also quite brief, but pointing towards lots of existing work around
 - introducing people to programming through music
 - introducing people to music through programming

 This section is primarily about where education, music and programming come together.

 

\section{WHY: why is music of any value in a programming and education context}

% (from the pitch)
Music and programming have a rich intertwined history. Simple-but-effective connections can be found between programming concepts (iteration, abstraction, conditionals, loops) and musical concepts (timbre, rhythm, timing, harmony, pattern). Writing a program that generates sound provides an immediate, engaging way to manifest abstract processes; in a pedagogical sense, the sound output is a form of feedback. Programs can often be short and simple but nevertheless very musically satisfying. 

Some relevant references:
\begin{enumerate}
\item Creativity in CS1: A Literature Review \cite{Sharmin2021}
\item Sonic Pi – performance in education, technology and art \cite{Aaron2016}
\end{enumerate}

It's not abstract! You get something at the end of it that you can show people.

(some useful lines in the pitch on this)

interesting parallels with music and code anyway: scores are instructions too.

Rhythm and pitch - tailor made for iteration, conditionals, etc.

Music to teach/engage with simulation (Mike, Charlotte). Fourier transform as a nice transferable example...

Examples:
\begin{enumerate}
    \item Fourier transform
    \item Wave interference
    \item Resonance cavities $\leftrightarrow{}$ Guitars
    \item Oscillators, coupled oscillators (Idea of pendulums that make sound. Need to increase corresponding frequency to make it audible.)
\end{enumerate}

\section{Overview of our online resources}
What are they, who are they for, what kinds of programming concepts do they relate to?

\section{Mini summary}
Close it with a quick summary

