\section*{[Title Goes Here!]}

\subsection*{Authors}
Tom Mudd, Charlotte Desvages, Joe Hathway, Mike Taverne, Matthew Hamilton, William Kay, Yashique Chalil, Alan O'Callaghan, student contributor(s) TBC

\subsection*{Roles}
Each author brings a different perspective on programming pedagogy and how music can play a part --- as (current and former) students, lecturers, tutors, learning technicians, musicians, sound designers...

\subsection*{Format}
The chapter will introduce links between music and programming, give an overview of relevant pedagogical literature on teaching programming in creative contexts and/or with multimedia applications, and provide insights on building programming tasks for students involving music or sound which are pedagogically valuable. An example task will be given within the chapter itself; a wider set of examples will be provided on GitHub as additional resources (roughly one per author).

\subsection*{Pitch}
Music and programming have a rich [combined/joint/intertwined] history. Music and audio programming naturally lends itself to teaching computer programming concepts (iteration, abstraction, conditionals...) as it applies in isolation or simultaneously to timbre, rhythm, harmony, etc. Furthermore, writing a program that generates sound provides a very immediate, engaging way to manifest abstract processes --- in a pedagogical sense, the sound output is a form of feedback. Programs can often be very simple (e.g. 2-3 lines) but nevertheless very musically satisfying. 

Students can be highly motivated to create and share the things they make, things that can be understood by friends and relatives, who may not know the first thing about programming but can nonetheless engage with the creative outcomes. Presenting technical tasks within a context that is familiar to the student has also been shown to improve [educational outcomes/achievement/understanding - I need a reference for this one].

The chapter is designed as an entry point for educators who may have no experience of music inside or outside or programming. The centrepiece of the chapter is a set of example tasks, challenges, or assignments for students that engage them with fundamental programming concepts through music making. Each example highlights one or more programming concepts and shows how music allows for a tangible engagement with that concept. The examples themselves are primarily digital resources: a mixture of recipes, code examples, audio examples, and other supporting media where relevant (e.g. images, diagrams, video demos). Each example is first presented as a blueprint for teachers, outlining the task and programming concepts covered, the learning outcomes, and suggesting appropriate levels of programming experience for students. Developed examples are then given in specific programming languages, that can be deployed directly, or extended, tweaked, and otherwise adapted to suit different situations. Suggestions are given for adaptations to other creative and technical applications outside of music and audio, following the same educational principles.

The chapter also serves as a gateway to the wider literature surrounding the topic, helping to orient readers who are new to the topic within the existing landscape of work in this area.

Once a draft is in place a small number of postgraduate students will be presented with it and their reflections, suggestions and contributions will be considered (with contributing students being included as co-authors).
