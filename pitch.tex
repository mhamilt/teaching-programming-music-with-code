\section*{[Title Goes Here!]}

\subsection*{Authors}
Tom Mudd, Charlotte Desvages, Joe Hathway, Mike Taverne, Matthew Hamilton, William Kay, Yash Chalil, Alan O'Callaghan, student contributor(s) TBC

\subsection*{Roles}
Each author brings a different perspective on programming pedagogy and how music can play a part.  

\subsection*{Format}
A set of worked examples as a github resource (roughly one per author). The chapter itself will introduce music and programming more generally.

\subsection*{Pitch}
Music has a rich history within programming. Having a program create sounding output provides a very immediate, engaging way to manifest abstract processes. Programs can often be very simple (e.g. 2-3 lines) but nevertheless very musically satisfying. 

Students can be highly motivated to create and share the things they make, things that can be understood by friends and relatives, who may not know the first thing about programming but can nonetheless engage with the creative outcomes.

The chapter is designed as an entry point for educators who may have no experience of music inside or outside or programming. The centrepiece of the chapter is a set of example tasks, challenges or assignments for students that engage them with programming concepts through music making. Each example highlights one or more programming concepts and shows how music allows for a tangible engagement with that concept. The examples themselves are primarily digital resources: a mixture of recipes, code examples, audio examples, and other supporting media where relevant (e.g. images, diagrams, video demos). Each example is presented in a generalised form as a recipe. They are then also presented in a more concrete fashion in specific languages that can be deployed directly, or extended, tweaked, and otherwise adapted to suit different situations.

The chapter also serves as a gateway to the wider literature surrounding the topic, helping to orient readers who are new to the topic within the existing landscape of work in this area.

Once a draft is in place a small number of postgraduate students will be presented with it and their reflections, suggestions and contributions will be considered (with contributing students being included as co-authors).
